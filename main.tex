% Это основная команда, с которой начинается любой \LaTeX-файл. Она отвечает за тип документа, с которым связаны основные правил оформления текста.
\documentclass{article}

% Здесь идет преамбула документа, тут пишутся команды, которые настраивают LaTeX окружение, подключаете внешние пакеты, определяете свои команды и окружения. В данном случае я это делаю в отдельных файлах, а тут подключаю эти файлы.

% Здесь я подключаю разные стилевые пакеты. Например возможности набирать особые символы или возможность компилировать русский текст. Подробное описание внутри.
\usepackage{packages}

% Здесь я определяю разные окружения, например, теоремы, определения, замечания и так далее. У этих окружений разные стили оформления, кроме того, эти окружения могут быть нумерованными или нет. Все подробно объяснено внутри.
\usepackage{environments}

% Здесь я определяю разные команды, которых нет в LaTeX, но мне нужны, например, команда \tr для обозначения следа матрицы. Или я переопределяю LaTeX команды, которые работают не так, как мне хотелось бы. Типичный пример мнимая и вещественная часть комплексного числа \Im, \Re. В оригинале они выглядят не так, как мы привыкли. Кроме того, \Im еще используется и для обозначения образа линейного отображения. Подробнее описано внутри.
\usepackage{commands}

% Пакет для титульника проекта
\usepackage{titlepage}

% Пакет для кода
\usepackage{listings}

% Здесь задаем параметры титульной страницы
\setUDK{004.942}
% Выбрать одно из двух
% \setToResearch
\setToProgram

\setTitle{Текстовый редактор с совместным редактированием статей и умным поиском}

% Выбрать одно из трех:
% КТ1 -- \setStageOne
% КТ2 -- \setStageTwo
% Финальная версия -- \setStageFinal
% \setStageOne
%\setStageTwo
%\setStageFinal

\setGroup{209}
%сюда можно воткнуть картинку подписи
\setStudentSgn{}

\setStudentDate{17.05.2022}
\setAdvisor{Казаков Евгений Александрович}
\setAdvisorTitle{разработчик}
\setAdvisorAffiliation{Facebook inc.}
\setAdvisorDate{17.05}
\setGrade{10}
%сюда можно воткнуть картинку подписи
\setAdvisorSgn{}
\setYear{2022}


% С этого момента начинается текст документа
\begin{document}

% Эта команда создает титульную страницу
\makeTitlePage

% Данное окружение оформляет аннотацию: краткое описание текста выделенным абзацем после заголовка
\begin{abstract}
какой-то текст
\end{abstract}

% Здесь будет автоматически генерироваться содержание документа
\tableofcontents

\input{sections/key_words.tex}

\section{Введение}

\subsection{Актуальность проблемы}
Сервис, предоставляющий поиск с использованием машинного обучения по документам, 
в настоящее время является актуальным и востребованным, 
так как современные организации и предприятия сталкиваются с большим объемом информации и необходимостью быстрого и эффективного ее поиска. 
Использование машинного обучения позволяет улучшить точность и скорость поиска, 
что является ключевым преимуществом для пользователей этого сервиса. 
В целом, актуальность сервиса, предоставляющего поиск с использованием машинного обучения по загруженным документам, 
связана с возрастающей необходимостью быстрого и эффективного поиска информации в современном мире,
а также с возможностью использования технологий машинного обучения для улучшения результатов поиска и анализа данных.


\subsection{Цели и задачи}
\subsubsection{Цель}
тут цель
\subsubsection{Задачи}
\begin{itemize}
\item 1-ая задача
\end{itemize}
конец
\subsection{Аналоги}

Преимущества Notion:

Удобный и красивый интерфейс 
Удобный редактор

Недостатки Notion:

Нельзя хранить информацию локально
Нет хорошего поиска

Преимущества Wiki:

Можно хостить локально

Недостатки Wiki:

Неудобный интерфейс 
Неудобный редактор
Нет хорошего поиска


Преимущества Confluence:

Удобный интерфейс 
Удобный редактор 
Можно хостить локально

Недостатки Confluence:

Нет хорошего поиска


основной 
\subsection{Описание функциональных требований к программному проекту}

\subsection{Описание нефункциональных требований к программному проекту}

\section{Предлагаемые подходы и методы}
Будем хранить векторное представление наших документов, которое отображает их в пространство признаков. 
Эти векторы будем создаавать при помощи нейронных сетей 
или алгоритмов обработки естественного языка, таких как BERT или TF-IDF, в зависимости от того, что будет лучше работать.

Для поиска будем использовать две основные компоненты - Reader и Retriever. 

Retriever - отвечает за извлечение подмножества документов, которые наиболее вероятно будут содержать информацию, 
необходимую для ответа на поисковый запрос. Retriever не осуществляет анализ содержимого документов, 
а основывается на индексировании и быстром поиске по ключевым словам.
В качестве алгоритмов индексации могут использоваться BM25, TF-IDF и другие методы.

Reader - отвечает за поиск ответов в найденных документах. Reader обрабатывает каждый документ, 
используя алгоритмы машинного обучения, чтобы найти наиболее релевантную информацию для поискового запроса. 
В частности, Reader может использовать нейронные сети и методы обработки естественного языка, 
чтобы извлекать ответы на вопросы из текстовых документов.

\section{Заключение}
\subsection{Результаты}
\subsubsection{Коннекторы}
Для взаимодействия с биржей был написан класс, использующий API. Класс предоставляет множество различных методов для алгоритмического трейдинга и может быть использован во многих проектах похожей направленности.
\subsubsection{Market Making}
Попробовали реализовать стратегию маркет-мейкинга, на которой мы не смогли заработать в основном из-за того, что на бирже \texttt{Dydx} очень мало сделок, порядка одной сделки в 3-4 секунды. Соответственно, рынок не высоко-инерциальный, а стратегия маркет-мейкинга работат в предположении высокой инерциальности.
\subsubsection{Торговля на основе индикаторов}
Была написана стратегия торговли, базирующаяся на информации об индикаторах. На основе информации о них с помощью библиотеки катбуст мы попытались предсказывать направление движения рынка. Наша стратегия не очень сработала, так как классы были трудно различимы.
\subsubsection{Арбитраж}
Предприняли попытку реализовать стратегию, которая торгует исходя из курса на другой бирже. Для этого написали инструмент, позволяющий рассчитать теоретическую прибыль. Попробовали отделять теоретически прибыльные сигналы от убыточных. 

Написали трейдера для этой стратегии, он слушает вебсокет Binance и торгует на dydx. Вот что-что, а трейдер получился не плохим, очень идиоматично написан, использует немного ресурсов и прост в понимании. Вот бы еще и стратегия работала.

\subsection{Дальнейшие перспективы}
\subsubsection{Разработка инфраструктуры для сбора информации}
По ходу написания и тестирования различных стратегий нами было написано множество различных инструментов для сбора и анализа, которые можно развить в полноценную библиотеку для сбора и анализа данных информации с основных бирж.
\subsubsection{Улучшение имеющихся стратегий}
Можно попробовать доделать уже имеющиеся стратегии. Существует очень много вариантов апгрейда наших наработок. Рассмотрим каждую из стратегий.
\begin{enumerate}
    \item \textbf{Market Making}
    \item \textbf{Индикаторы}
    
    Можно придумать новые индикаторы, которые будут использоваться при классификации. Можно обратиться к уже имеющимся вариантам и подыскать там что-то подходящее. Рассмотрение других индикаторов, возможно, приведет к лучшему отделению классов друг от друга. 
    
    Также не стоит забывать о большом количестве моделей для классификации, которые можно использовать для поставленных задач. Быть может, какая-то из них даст нам лучший результат.
    
    В нашей реализации мы никак не использовали данные со стакана. Дальнейшее развитие может заключаться именно в этом: как-то анализировать не только совершенные сделки, но и смотреть, как ведут себя все участники рынка.
    \item \textbf{Арбитраж}
    
    В этой стратегии стоит работать в направлении поиска новых фичей, которые могли бы помочь отделить теоретические успешные сигналы от провальных. Также можно посмотреть в сторону разработки инструмента для подбора ключевых параметров стратегии. Можно слушать несколько бирж и пробовать каким-то образом усреднять сигналы. 
    
    Следует оптимизировать трейдер, возможно, переписать его на другой, более быстрый язык программирования, ведь здесь нам важна каждая миллисекунда.
    
    Попытаться найти географическое положение, в котором задержка до сигнальной биржи и до той, на которой мы торгуем, будет минимальная. В этой стратегии такой характер задержки носит принципиальный характер.
\end{enumerate}

\subsubsection{Поиск новых бирж и инструментов}
Конечно, всегда можно попытаться найти какую-то свежую биржу, на которой еще не расплодились боты всяких HFT фондов, и попробовать уже написанные стратегии. Вполне вероятно, что какое-то время они могут приносить деньги. Главное вовремя понять, когда это закончиться. И следует обратить внимание на другие инструменты и валюты, которые мы обошли стороной. На них успех тоже возможен.

\subsubsection{Переход к по-настоящему децентрализованным технологиям}
Можно направить проект немного в другое русло и рассмотреть торговлю непосредственно в сети эфира. Это уже влечет написание смарт-контрактов и тому подобное. Надо основательно разобраться в блокчейне и изучить технологии. Но тогда нам откроется множество новых подходов к алгоритмическому трейдингу. Они могут быть крайне прибыльными.
    


\section{Дополнительные результаты}

Помимо задач, которые нам поставил руководитель, мы сделали еще:

\subsection{Визуализация}
Перед тем, как работать с данными, надо понять, как они устроены: попарное распределение классов, плотность каждого из индикаторов. Мы все это сделали. Теперь стало гораздо проще подбирать параметры для модели машинного обучения.

\subsection{Контролирующий бот}
Мы сделали телеграм бота, на которого можно по паролю подписаться и получать обновления состояния аккаунта на бирже. Это полезно, когда ты запускаешь торговую стратегию, куда-то отходишь, но при этом всегда можешь контролировать, что с ней происходит, через телеграм.

\subsection{CI/CD, тесты, линтер}
Любая ошибка в торговой системе может потерять наши деньги, поэтому важно, чтобы весь код всегда был рабочим. Для этого мы все, что смогли, обложили тестами c использованием утилиты PyTest~\cite{Pytest}. Теперь при каждом пулл реквесте у нас запускается тестирующая система, и если какие-то тесты не проходят, то мы запрещаем дальнейший пуш. Реализовали мы это через GitHub Actions~\cite{GitHubActions}.

Еще нам хочется, чтобы весь код был консистеным. Для этого мы используем линтер Black~\cite{Black} и статический анализатор PyLint~\cite{Pylint}. В совокупности эти утилиты поддерживают консистентность нашего кода и могут еще до запуска теста, выдать синтаксические ошибки.

\subsection{Смарт контракты}
С помощью смарт контрактов можно занимать у других людей миллиарды, трейдить на них, и потом возвращать криптовалюту обратно. Звучит заманчиво. Мы написали смарт конракты на Solidity~\cite{Solidity}, и залили их в сеть эфира через Brownie~\cite{Brownie}.

% там как-то надо красиво оформлять списки статей 

%про алгоритм BM25
http://www.staff.city.ac.uk/~sbrp622/papers/foundations_bm25_review.pdf

%про трансформер BERT
https://arxiv.org/abs/1810.04805



% Здесь автоматически генерируется библиография. Первая команда задает стиль оформления библиографии, а вторая указывает на имя файла с расширением bib, в котором находится информация об источниках.
\bibliographystyle{plainurl}
\bibliography{bibl}

% Здесь текст документа заканчивается
\end{document}
% Начиная с этого момента весь текст LaTeX игнорирует, можете вставлять любую абракадабру.
