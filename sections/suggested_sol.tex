\section{Предлагаемые подходы и методы}
Будем хранить векторное представление наших документов, которое отображает их в пространство признаков. 
Эти векторы будем создаавать при помощи нейронных сетей 
или алгоритмов обработки естественного языка, таких как BERT или TF-IDF, в зависимости от того, что будет лучше работать.

Для поиска будем использовать две основные компоненты - Reader и Retriever. 

Retriever - отвечает за извлечение подмножества документов, которые наиболее вероятно будут содержать информацию, 
необходимую для ответа на поисковый запрос. Retriever не осуществляет анализ содержимого документов, 
а основывается на индексировании и быстром поиске по ключевым словам.
В качестве алгоритмов индексации могут использоваться BM25, TF-IDF и другие методы.

Reader - отвечает за поиск ответов в найденных документах. Reader обрабатывает каждый документ, 
используя алгоритмы машинного обучения, чтобы найти наиболее релевантную информацию для поискового запроса. 
В частности, Reader может использовать нейронные сети и методы обработки естественного языка, 
чтобы извлекать ответы на вопросы из текстовых документов.