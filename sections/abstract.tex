Говорить о криптовалюте можно много, но интереснее знать, как она устроена изнутри и как на ней можно заработать. Эти цели мы поставили в своей работе.


Результатами стали:

\begin{enumerate}

\item Коннекторы для подключения к биржам, чтобы было просто с ними взамодействовать через API.

\item Маркет-мейкинг стратегия.

\item Модель на основе решающих деревьев и градиентного бустинга, предсказывающая направление рынка, базирующаяся на информации об индикаторах.

\item Арбитражная стратегия, которая торгует исходя из курса на другой бирже.

\end{enumerate}


Пайплан нашей работы был такой:

\begin{enumerate}
\item Сначала мы искали торговые стратегии. Рассматривали как самые современные, так и достаточно старые, которые можно переосмыслить с современными технологиями.
\item Потом мы собирали данные для экспериментов. Для этого написали функцию в коннекторе, куда можно подать интересующий интервал времени, и она выдаст датасет с историческими данными.
\item Далее мы очищали данные от мусора, чтобы эксперимент сделать как можно точнее
\item И на самом интересном шаге -- эксперименте -- мы прототипировали нашу стратегию и прогоняли на исторических данных, чтобы понять, можно ли что-то заработать или нет.
\item На последнем шаге мы создавали трейдер. Именно он отвечает за непосредственную торговлю на бирже. Оптимизировали стратегию, чтобы она работала как можно быстрее и занимала как можно меньше памяти. Нам было важно писать код так, чтобы его можно было переиспользовать в других стратегиях, и не делать одну и ту же работу несколько раз.
\end{enumerate}

