\subsection{Измерение задержки, скорости работы коннектора}

Нам очень важно понимать, как быстро работает взаимодействие с биржей по средствам нашего коннектора. Для анализа возможной прибыли от различных стратегий просто необходимо знать сколько, например, занимает отправка рыночного ордера: от момента отправки запроса с локальной машины до момента, когда он встал в очередь на исполнение.

Для этих целей был написан отдельный \href{https://github.com/dexety/dex-trading-system/blob/90a2ccd08234584b2ab9a274fe82533a55b924d9/utils/dydx_delay_measurer.py#L15}{\texttt{class SpeedMeasure}}, который измеряет скорость работы \texttt{DydxConnector}.  С помощью него можно получить время работы всех методов коннектора, а также время, которое занимает, чтобы ордер встал в очередь на исполнение. На выходе получаются два файла, которые содержат следующую информацию по обоим пунктам: самое быстрое время, самое медленное время, среднее время. Вот пример содержания этих файлов:

\begin{verbatim}
[
    {
        "meta": {
            "datetime": "2022-05-18 08:31:44.567831",
            "symbol": "ETH-USD",
            "side": "BUY",
            "orders_num": 5
        },
        "results": {
            "from_send_to_pending": {
                "average": 0.33753323554992676,
                "slowest": 0.33753323554992676,
                "fastest": 0.33753323554992676
            },
            "from_pending_to_open": {
                "average": 0.025999784469604492,
                "slowest": 0.025999784469604492,
                "fastest": 0.025999784469604492
            },
            "from_our_cancel_to_serv_cancel": {
                "average": 0.10059189796447754,
                "slowest": 0.10059189796447754,
                "fastest": 0.10059189796447754
            }
        }
    }
]
\end{verbatim}
И аналогичный файл создается для методов класса \texttt{DydxConnector}, в нем присутствует информация для каждой функции в классе:
\begin{verbatim}
[
    {
        "meta": {
            "datetime": "2022-05-18 08:12:26.824202",
            "symbol": "ETH-USD",
            "side": "BUY",
            "iters_num": 10
        },
        "results": {
            "get_user": {
                "average": 0.161401629447937,
                "fastest": 0.1470780372619629,
                "slowest": 0.22996306419372559
            },
            ...
\end{verbatim}
Это дает полное понимание о скорости работы нашего коннектора.

Исходя из информации о том, где находятся сервера биржи dydx и методом научного тыка, мы выяснили, что на серверах AWS в Вирджинии время на отправку ордера минимально. Поэтому, если дойдет до полноценного разворота торговой системы, это будет происходить именно там. Такие цифры мы получили: 
\begin{verbatim}
Со своей машины в России
0.3860666275024414 секунд во время слабой нагрузки на dydx
0.501491904258728 секунд во время сильной нагрузки на dydx
С АВС в штатах
0.13387751579284668 секунд во время слабой нагрузки на dydx
0.40620856285095214 секунд во время сильной нагрузки на dydx
\end{verbatim}

Задержку до биржи Binance мы измеряли только на отправку нам трейдов. Делали так: Бинанс при отправке трейда по вебсокету проставляет время, когда он был отправлен, мы, получив его на локальной машине, вычитали время, отмеченное Бинансом, из текущего и получали задержку. Она составляла в районе 200-300мс максимум. Вполне приемлемо.