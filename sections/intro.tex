\section{Введение}

\subsection{Актуальность проблемы}
С постоянным увеличением объема данных и информации, с которыми мы сталкиваемся в повседневной жизни, 
поиск информации вручную становится все более трудоемким и неэффективным. 
Поиск по документам при помощи машинного обучения позволяет автоматизировать процесс поиска, облегчить его и сэкономить время.

С помощью машинного обучения, сервис может находить и анализировать не только ключевые слова, но и контекст,
смысл и связи между документами. Это значительно повышает точность и эффективность поиска, 
а также позволяет находить связанные документы и информацию.

Кроме того, использование машинного обучения также позволяет сервису
улучшать свои алгоритмы и становиться более точными с течением времени, что делает их все более востребованными.

Таким образом, поиск по документам при помощи машинного обучения является
актуальным и полезным в наше время, позволяя нам быстро и эффективно находить нужную информацию в огромном объеме данных.


\subsection{Цели и задачи}
\subsubsection{Цель}
Разработать сервис тексового редактора с функцией поиска по документам при помощи машинного обучения
\subsubsection{Задачи}
\begin{itemize}
\item % какие-то задачи про текстовый редактор
\item найти алгоритм поиска документа, наиболее подходящего под поступающий вопрос
\item найти подходящую модель для поиска ответа по контексту 
\item реализовать сервис поиска ответа по вопросу в документах
\end{itemize}
конец
\subsection{Аналоги}

Преимущества Notion:

Интуитивно понятный и простой в использовании интерфейс
Удобный и многофункциональный редактор

Недостатки Notion:

Нельзя хранить информацию локально
Нет хорошего поиска
Нет системы доступов - все, кто имеют доступ к документу, могут его редактировать

Преимущества Wiki:

Можно хостить локально

Недостатки Wiki:

Неудобный интерфейс 
Неудобный редактор
Нет хорошего поиска


Преимущества Confluence:

Удобный интерфейс 
Удобный редактор 
Можно хостить локально

Недостатки Confluence:

Нет хорошего поиска


основной 
\subsection{Описание функциональных требований к программному проекту}
Должен быть класс, который реализует поиск по документам. У него должны быть следующие функций:
\begin{itemize}
    \item обновление статьи 
    \item добавление статьи
    \item ответ на вопрос
\end{itemize}

\subsection{Описание нефункциональных требований к программному проекту}